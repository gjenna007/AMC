1) Het specklegroeien zelf versnellen

De oorzaak van het traag lopen van het programma (ook in C) zit hem erin dat je voor ieder nieuwe speckle die je gaat groeiten het hele beeld gaat doorlopen, om weer een nieuw maximum te vinden. Zo ga je het beeld honderden keren moeten doorlopen. Dit kun je echter op de volgende manier versnellen:

Je bepaalt van iedere pixel de gijrswaarde, en die grijswaarden met daaraan gekoppeld de (coordinaten van de) pixels ga je sorteren op grijswaarde. Zo kun je de masima vinden door die lijst van grijswaarden van boven naar beneden een voor een af te werken. 
Als je echter een speckle groeit, moet je op die plaatsen van die coordianten in de tabel een ``void'' waarde zetten (dat als daar ``aankomt'' om die als volgende zaad voor het groeien te nemen te kunnnen zien dat die al in gebruik is). Hiervoor moet je echter de tabel van de coordinaten weer helemaal doorlopen, want die is niet gesorteerd. Dit is eigenlijk het hart van het probleem: twee gekoppelde lijsten (een met de grijswaarden en een met de coordinaten) waarvan je of de een of de andere kunt sorteren.

Het opzoeken van de coordinaten kun je echter versnellen de grijswaarde te gaan gebruiken als een ``bookmark'' vanaf waar je moet gaan zoeken in de tabel.
Je houdt in een derde klein tabelletje bij op welke plaats in de tabel welke grijswaarde begint (dus grijswaarde 255 op plaats 1, 254 op plaats ...). Je knipt de tabel aan de hand van de grijswaarde dus als het ware in 256 stukken.
Je kunt de tabel nu nog als het ware verder op knippen door meer dan 1 kenmerk (de grijswaarde) te gaan gebruiken. Je kunt bijvoorbeeld in 4 of 16 stukken verdelden, en op die manier de plaats als tweede n=bookmark introduceren.
Hiervoor moet je dus eerste de tabel sorteren naar grijswaarden, en dan alle pixels met dezelfde grijswaarde weer verder sorteren naar het vak waarin ze zitten. De tabel met bookmarks wordt nu wel weer groter. Als je nu een pixel kleurt in het beeld (en dus moet uitschakelen in de tabel), kun je in het beeld meteen bepalen welke grijswaade die pixel heeft, en in welk vak die zit. In je bookmarkstabel zoek je nu op vanaf welke plaats in de gesorteerde grijswaarden je moet gaan zoeken, en dan zoek je in de coordinatentabel de juiste coordinaten op ( van die pixel) en schakelt hem uit. De gijswaardentabel laat je onaangeroerd.

Je begrijpt dat als je het scherm maar in een groot genoeg aantal vakken opdeelt je ieder pixel afzonderlijk gewoon zit te klassificeren. Er zal dus een optimale middenweg zijn tussen het hele beeld beschouwen als 1 vlak, en iedere pixel apart beschouwen als een vlak.

2) Het region growen moet je m.b.v. een recursieve functie aanroep doen!!
De punten van waaruit je gaat groeien voeren we in als een apart beeld. Zo kunnen we het zoeken van toppen etc aan matlab overlaten.

Misschien is dit alles wel een heel slecht idee.